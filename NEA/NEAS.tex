\documentclass[a4paper,12pt]{article}
\usepackage{fancyhdr}
\usepackage{fancyheadings}
\usepackage[ngerman,german]{babel}
\usepackage{german}
\usepackage[utf8]{inputenc}
%\usepackage[latin1]{inputenc}
\usepackage[active]{srcltx}
\usepackage{algorithm}
\usepackage[noend]{algorithmic}
\usepackage{amsmath}
\usepackage{amssymb}
\usepackage{amsthm}
\usepackage{bbm}
\usepackage{enumerate}
\usepackage{graphicx}
\usepackage{ifthen}
\usepackage{listings}
\usepackage{struktex}
\usepackage{hyperref}
\usepackage{tikz}
\usetikzlibrary{positioning,automata}

%%%%%%%%%%%%%%%%%%%%%%%%%%%%%%%%%%%%%%%%%%%%%%%%%%%%%%
%%%%%%%%%%%%%% EDIT THIS PART %%%%%%%%%%%%%%%%%%%%%%%%
%%%%%%%%%%%%%%%%%%%%%%%%%%%%%%%%%%%%%%%%%%%%%%%%%%%%%%
\newcommand{\Fach}{Theoretische Informatik}
\newcommand{\Name}{Robin Feldmann}
\newcommand{\Tutorium}{}
\newcommand{\Semester}{SS 21}
\newcommand{\KlausurLoesung }{SoSe2020} %  <-- UPDATE ME
%%%%%%%%%%%%%%%%%%%%%%%%%%%%%%%%%%%%%%%%%%%%%%%%%%%%%%
%%%%%%%%%%%%%%%%%%%%%%%%%%%%%%%%%%%%%%%%%%%%%%%%%%%%%%

\setlength{\parindent}{0em}
\topmargin -1.0cm
\oddsidemargin 0cm
\evensidemargin 0cm
\setlength{\textheight}{9.2in}
\setlength{\textwidth}{6.0in}

%%%%%%%%%%%%%%%
%% Aufgaben-COMMAND
\newcommand{\Aufgabe}[1]{
  {
  \vspace*{0.5cm}
  \textsf{\textbf{Aufgabe #1}}
  \vspace*{0.2cm}
  
  }
}

\newcommand{\Definition}[1]{
  {
  \vspace*{0.5cm}
  \textsf{\textbf{Definition #1}}
  \vspace*{0.2cm}
  
  }
}

%%%%%%%%%%%%%%
\hypersetup{
    pdftitle={\Fach{}: Übungsblatt \KlausurLoesung{}},
    pdfauthor={\Name},
    pdfborder={0 0 0}
}

\lstset{ %
language=java,
basicstyle=\footnotesize\tt,
showtabs=false,
tabsize=2,
captionpos=b,
breaklines=true,
extendedchars=true,
showstringspaces=false,
flexiblecolumns=true,
}

\title{Übungsblatt Deterministische Endliche Automaten}
\author{\Name{}}

\begin{document}
\thispagestyle{fancy}
\lhead{\sf \large \Fach{} \\ \small \Name{} }
\rhead{\sf \Semester{} \\  Tutorium \Tutorium{}}
\vspace*{0.2cm}
\begin{center}
\LARGE \sf \textbf{Nichtdeterministische Endliche Automaten}
\end{center}
\vspace*{0.2cm}

%%%%%%%%%%%%%%%%%%%%%%%%%%%%%%%%%%%%%%%%%%%%%%%%%%%%%%
%% Insert your solutions here %%%%%%%%%%%%%%%%%%%%%%%%
%%%%%%%%%%%%%%%%%%%%%%%%%%%%%%%%%%%%%%%%%%%%%%%%%%%%%%
\Definition{Nichtdeterministischer Endlicher Automat}
Ein endlicher nichtdeterministischer Automat $A = (X,S,S_0,\delta,F)$ besteht aus:
\begin{align*}
X&: Endliches \ Eingabealphabet \\
S&: Endliche \  Zustandsmenge\\
S_0&: Menge\  der\ Startzust"ande \subseteq S \\
\delta &: Zustands"ubergangsfunktion: \delta : S \times X \rightarrow P(S) \\
F&: Menge\ der\ Endzust"ande \subseteq S\\
\end{align*}

\Aufgabe{1}
Gegeben sei das Alphabet $X = \{a,b,c\}$.

Konstruieren sie einen nichtdeterministischen endlichen Automaten der alle W"orter akzeptiert, welche die Zeichenkette acab enthalten. Geben sie den Automaten in Form eines "Ubergangsgraphen und eines Tupels an.
 


\Aufgabe{2}
Gegeben sei folgender nichtdeterministischer endlicher Automat.
$$ A = (\{a,b\},\{S_0,S_1,S_2\}, \{S_0\}, \delta \ gem"aß \ Graph, \{S_2\})$$

\begin{tikzpicture}[auto, thick, auto]
  \node[initial,state] (S0) {$S_0$};
  \node[state, right=of S0] (S1) {$S_1$};
   \node[state, right=of S1, accepting] (S2) {$S_2$};
\path[->]
(S0) edge [bend left] node {a} (S1)
(S1) edge [bend left] node {a} (S2)
(S1) edge [bend left] node {a,b} (S0);
\end{tikzpicture}

Konstruieren sie entsprechend des Beweises in der Vorlesung den zugeh"origen deterministischen endlichen Automaten. Geben sie also die Zustands"ubergangsfunktion f"ur jedes Element aus der Potenzmenge  $P(\{S_0,S_1,S_2\})$ an.

\Aufgabe{3}
\begin{enumerate}[a)]
\item
Gegeben sei folgender nichtdeterministischer endlicher Automat.
$$ A = (\{a,b\},\{S_0,S_1,S_2\}, \{S_0\}, \delta \ gem"aß \ Tabelle, \{S_2\})$$


\begin{tabular}{l|l|l}
$\delta$ & a & b  \\
\hline
$S_0$ & $S_1$ & $S_2$ \\
\hline
$S_1$ & $S_1,S_2$ & $\{ \}$ \\
\hline
$S_2$ & $\{ \}$& $\{ \}$ \\
\end{tabular}
\\
\\
Konstruieren sie den zugehörigen endlichen deterministischen Automaten.
Geben sie diesen als Tupel und die Zustands"ubergangsfunktion als Tabelle an.

\item
Gegeben sei folgender nichtdeterministischer endlicher Automat.
$$ A = (\{a,b\},\{S_0,S_1,S_2,S_3\}, \{S_0,S_3\}, \delta \ gem"aß \ Tabelle, \{S_2,S_3\})$$


\begin{tabular}{l|l|l}
$\delta$ & a & b  \\
\hline
$S_0$ & $S_0, S_1$ & $\{ \}$ \\
\hline
$S_1$ & $\{ \}$ & $S_1,S_2$ \\
\hline
$S_2$ & $S_0, S_2$& $\{ \}$ \\
\hline
$S_3$ & $\{ \}$& $S_1$ \\
\end{tabular}
\\
\\
Konstruieren sie den zugehörigen endlichen deterministischen Automaten.
Geben sie diesen als Tupel und die Zustands"ubergangsfunktion als Tabelle an.
\end{enumerate}


\Aufgabe{4}
Gegeben sei folgender nichtdeterministischer endlicher Automat.
$$ A = (\{x,y\},\{N_0,N_1,N_2,N_3\}, \{N_0,N_2\}, \delta \ gem"aß \ Graph, \{N_1,N_3\})$$


\begin{tikzpicture}[auto, thick, auto]
  \node[initial,state] (N0) {$N_0$};
  \node[state, right=of N0, accepting] (N1) {$N_1$};
\path[->]
(N0) edge [bend left] node {x} (N1)
(N1) edge [loop above] node {y} (N1);
\end{tikzpicture}

\begin{tikzpicture}[auto, thick, auto]
   \node[initial, state] (N2) {$N_2$};
     \node[state, right=of N2, accepting] (N3) {$N_3$};
\path[->]
(N2) edge [bend left] node {y} (N3)
(N2) edge [loop above] node {x} (N2);
\end{tikzpicture}
\\
\\
Konstruieren sie den zugehörigen endlichen deterministischen Automaten.
Geben sie diesen als Tupel und die Zustands"ubergangsfunktion als Tabelle an (Angelehnt an SoSe17 Aufgabe 1. b)) .






%%%%%%%%%%%%%%%%%%%%%%%%%%%%%%%%%%%%%%%%%%%%%%%%%%%%%%
%%%%%%%%%%%%%%%%%%%%%%%%%%%%%%%%%%%%%%%%%%%%%%%%%%%%%%
\end{document}

