\documentclass[a4paper,12pt]{article}
\usepackage{fancyhdr}
\usepackage{fancyheadings}
\usepackage[ngerman,german]{babel}
\usepackage{german}
\usepackage[utf8]{inputenc}
%\usepackage[latin1]{inputenc}
\usepackage[active]{srcltx}
\usepackage{algorithm}
\usepackage[noend]{algorithmic}
\usepackage{amsmath}
\usepackage{amssymb}
\usepackage{amsthm}
\usepackage{bbm}
\usepackage{enumerate}
\usepackage{graphicx}
\usepackage{ifthen}
\usepackage{listings}
\usepackage{struktex}
\usepackage{hyperref}
\usepackage{tikz}
\usetikzlibrary{positioning,automata}

%%%%%%%%%%%%%%%%%%%%%%%%%%%%%%%%%%%%%%%%%%%%%%%%%%%%%%
%%%%%%%%%%%%%% EDIT THIS PART %%%%%%%%%%%%%%%%%%%%%%%%
%%%%%%%%%%%%%%%%%%%%%%%%%%%%%%%%%%%%%%%%%%%%%%%%%%%%%%
\newcommand{\Fach}{Theoretische Informatik}
\newcommand{\Name}{Robin Feldmann}
\newcommand{\Tutorium}{}
\newcommand{\Semester}{OHM}
\newcommand{\KlausurLoesung }{SoSe2020} %  <-- UPDATE ME
%%%%%%%%%%%%%%%%%%%%%%%%%%%%%%%%%%%%%%%%%%%%%%%%%%%%%%
%%%%%%%%%%%%%%%%%%%%%%%%%%%%%%%%%%%%%%%%%%%%%%%%%%%%%%

\setlength{\parindent}{0em}
\topmargin -1.0cm
\oddsidemargin 0cm
\evensidemargin 0cm
\setlength{\textheight}{9.2in}
\setlength{\textwidth}{6.0in}

%%%%%%%%%%%%%%%
%% Aufgaben-COMMAND
\newcommand{\Aufgabe}[1]{
  {
  \vspace*{0.5cm}
  \textsf{\textbf{Aufgabe #1}}
  \vspace*{0.2cm}
  
  }
}

\newcommand{\Definition}[1]{
  {
  \vspace*{0.5cm}
  \textsf{\textbf{Definition #1}}
  \vspace*{0.2cm}
  
  }
}

%%%%%%%%%%%%%%
\hypersetup{
    pdftitle={\Fach{}: Übungsblatt \KlausurLoesung{}},
    pdfauthor={\Name},
    pdfborder={0 0 0}
}

\lstset{ %
language=java,
basicstyle=\footnotesize\tt,
showtabs=false,
tabsize=2,
captionpos=b,
breaklines=true,
extendedchars=true,
showstringspaces=false,
flexiblecolumns=true,
}

\title{Übungsblatt Deterministische Endliche Automaten}
\author{\Name{}}

\begin{document}
\thispagestyle{fancy}
\lhead{\sf \large \Fach{} \\ \small \Name{} }
\rhead{\sf \Semester{} \\  Tutorium \Tutorium{}}
\vspace*{0.2cm}
\begin{center}
\LARGE \sf \textbf{Deterministische Endliche Automaten}
\end{center}
\vspace*{0.2cm}

%%%%%%%%%%%%%%%%%%%%%%%%%%%%%%%%%%%%%%%%%%%%%%%%%%%%%%
%% Insert your solutions here %%%%%%%%%%%%%%%%%%%%%%%%
%%%%%%%%%%%%%%%%%%%%%%%%%%%%%%%%%%%%%%%%%%%%%%%%%%%%%%
\Definition{Deterministischer Endlicher Automat}
Ein endlicher deterministischer Automat $A = (X,S,s_0,\delta,F)$ besteht aus:
\begin{align*}
X&: Endliches \ Eingabealphabet \\
S&: Endliche \  Zustandsmenge\\
s_0&: Startzustand \in S \\
\delta &: Zustands"ubergangsfunktion: \delta : S \times X \rightarrow S \\
F&: Menge\ der \ Endzust"ande \subseteq S\\
\end{align*}

\Aufgabe{1}
Gegeben sei das Alphabet $X = \{A,B,C\}$.

Konstruieren sie einen deterministischen endlichen Automaten der alle W"orter akzeptiert, welche die Zeichenkette ACAB enthalten. Geben sie den Automaten in Form eines "Ubergangsgraphen sowie in Form einer "Ubergangstabelle an. 
 



\Aufgabe{2} 
Geben sei das Alphabet $X =\{a,b\}$.
Geben sie für folgende Sprachen L einen endlichen deterministischen Automaten A an mit $L = L(A)$.
Mit L =
\begin{enumerate}[a)]
\item
$$\{ w \in X^* \mid \vert w \vert = 0\}$$
\item
$$\{ w \in X^* \mid \vert w \vert = 3 \}$$
\item
$$\{ w \in X^* \mid w = abw_1 \ mit \ w_1 \in X^*\}$$
\item
$$\{ w \in X^* \mid w = w_1ab \ mit \ w_1 \in X^*\}$$
\item
$$\{ w \in X^* \mid \vert w \vert _a = 3\}$$
\item
$$\{ w \in X^* \mid \vert w \vert _a \neq 3\}$$
\item
$$\{ w \in X^* \mid \vert w \vert _a = 2 \land  \vert w \vert _b = 1 \}$$
\item
$$\{ w \in X^* \mid \vert w \vert _a = 2 \lor  \vert w \vert _b = 1 \}$$
\item
$$\{ w \in X^* \mid w = (ab)^n(aabb)^m \ mit \ n\geq1, m\geq 1 \}$$
\item
$$\{ w \in X^* \mid w = ab^n, n \in \mathbb{N}, n\geq2 \} \cup \{ w \in X^* \mid w = ba^m, m \in \mathbb{N}, m\geq2 \}$$
\end{enumerate}

\Aufgabe{3}
Welche Sprache akzeptiert der dargestellte Automat? Geben sie diese in Mengenschreibweiße an.

\begin{tikzpicture}[auto, thick, auto]
  \node[initial,state] (S0) {$S_0$};
  \node[state, right=of S0] (S1) {$S_1$};
   \node[state, right=of S1] (S2) {$S_2$};
    \node[accepting, state, right=of S2] (S3) {$S_3$};
    \node[state, below= of S0] (S4) {$S_4$};
\path[->]
(S3) edge [loop above] node {b} (S3)
(S2) edge [loop above] node {b} (S2)
(S4) edge [loop below] node {a,b} (S4)
(S1) edge node {a} (S2)
(S2) [bend left]edge node {a} (S3)
(S0) edge node {b} (S4)
(S1) edge node {b} (S4)
(S0) edge node {a} (S1)
(S3) [bend left]edge node {a} (S2);
\end{tikzpicture}

\Aufgabe{4}

Konstruieren sie einen deterministischen endlichen Automaten "uber dem Alphabet $X = \{x,y,z\}$, der nur W"orter akzeptiert, die mit y beginnen und mit z enden (Angelehnt an SoSe20 Aufgabe 4).

\Aufgabe{5}

Konstruieren sie einen deterministischen endlichen Automaten der alle W"orter der Sprache 
$$  A = \{w \in \{x,y\}^* \mid  w = y^m x^n mit\  m,n \in \mathbb{N} \land \vert w \vert = ungerade \}$$
annimmt (angelehnt an SoSe20 Probeklausur Aufgabe 4).

%%%%%%%%%%%%%%%%%%%%%%%%%%%%%%%%%%%%%%%%%%%%%%%%%%%%%%
%%%%%%%%%%%%%%%%%%%%%%%%%%%%%%%%%%%%%%%%%%%%%%%%%%%%%%
\end{document}

