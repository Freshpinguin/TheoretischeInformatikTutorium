\documentclass[a4paper,12pt]{article}
\usepackage{fancyhdr}
\usepackage{fancyheadings}
\usepackage[ngerman,german]{babel}
\usepackage{german}
\usepackage[utf8]{inputenc}
%\usepackage[latin1]{inputenc}
\usepackage[active]{srcltx}
\usepackage{algorithm}
\usepackage[noend]{algorithmic}
\usepackage{amsmath}
\usepackage{amssymb}
\usepackage{amsthm}
\usepackage{bbm}
\usepackage{enumerate}
\usepackage{graphicx}
\usepackage{ifthen}
\usepackage{listings}
\usepackage{struktex}
\usepackage{hyperref}
\usepackage{tikz}
\usetikzlibrary{positioning,automata}

%%%%%%%%%%%%%%%%%%%%%%%%%%%%%%%%%%%%%%%%%%%%%%%%%%%%%%
%%%%%%%%%%%%%% EDIT THIS PART %%%%%%%%%%%%%%%%%%%%%%%%
%%%%%%%%%%%%%%%%%%%%%%%%%%%%%%%%%%%%%%%%%%%%%%%%%%%%%%
\newcommand{\Fach}{Theoretische Informatik}
\newcommand{\Name}{Robin Feldmann}
\newcommand{\Tutorium}{}
\newcommand{\Semester}{OHM}
\newcommand{\KlausurLoesung }{SoSe2020} %  <-- UPDATE ME
%%%%%%%%%%%%%%%%%%%%%%%%%%%%%%%%%%%%%%%%%%%%%%%%%%%%%%
%%%%%%%%%%%%%%%%%%%%%%%%%%%%%%%%%%%%%%%%%%%%%%%%%%%%%%

\setlength{\parindent}{0em}
\topmargin -1.0cm
\oddsidemargin 0cm
\evensidemargin 0cm
\setlength{\textheight}{9.2in}
\setlength{\textwidth}{6.0in}

%%%%%%%%%%%%%%%
%% Aufgaben-COMMAND
\newcommand{\Aufgabe}[1]{
  {
  \vspace*{0.5cm}
  \textsf{\textbf{Aufgabe #1}}
  \vspace*{0.2cm}
  
  }
}

\newcommand{\Definition}[1]{
  {
  \vspace*{0.5cm}
  \textsf{\textbf{Definition #1}}
  \vspace*{0.2cm}
  
  }
}

%%%%%%%%%%%%%%
\hypersetup{
    pdftitle={\Fach{}: Übungsblatt \KlausurLoesung{}},
    pdfauthor={\Name},
    pdfborder={0 0 0}
}

\lstset{ %
language=java,
basicstyle=\footnotesize\tt,
showtabs=false,
tabsize=2,
captionpos=b,
breaklines=true,
extendedchars=true,
showstringspaces=false,
flexiblecolumns=true,
}

\title{Übungsblatt Reguläre Sprachen}
\author{\Name{}}

\begin{document}
\thispagestyle{fancy}
\lhead{\sf \large \Fach{} \\ \small \Name{} }
\rhead{\sf \Semester{} \\  Tutorium \Tutorium{}}
\vspace*{0.2cm}
\begin{center}
\LARGE \sf \textbf{Reguläre Ausdrücke}
\end{center}
\vspace*{0.2cm}

%%%%%%%%%%%%%%%%%%%%%%%%%%%%%%%%%%%%%%%%%%%%%%%%%%%%%%
%% Insert your solutions here %%%%%%%%%%%%%%%%%%%%%%%%
%%%%%%%%%%%%%%%%%%%%%%%%%%%%%%%%%%%%%%%%%%%%%%%%%%%%%%


\Aufgabe{1 }
Gegeben sei die Sprache $ L \ = \ \{w \in \{a,b,c\}^* \mid \ w \  beginnt \ mit \ a \ und \ endet \ mit \ b\}. $ 
\begin{enumerate}[a)]
\item
Geben sie einen Regulären Ausdruck G an, sodass L(G) = L.
\item 
Geben sie einen deterministischen endlichen Automaten A an, sodass L(A) = L .
\end{enumerate}
(WiSe 19/20)
\\
\\
\Aufgabe{2}

Gegeben sei der Reguläre Ausdruck $ R = (x \cup \epsilon)\cdot (y \cup xy)^* $.
\begin{enumerate} [a)]
\item
Geben sie alle Wörter der Sprache $ L_3 = \{w \in L(R) \mid \vert w \vert \leq 3 \}$ explizit an. 
\item
Geben sie einen endlichen Automaten A an, sodass L(A) = L(R).
\end{enumerate}
(SoSe20)
\\
\\
\Aufgabe{3}
Gegeben sei der regulärer Ausdruck $R = ac^*b$. Konstruieren sie mit aus der Vorlesung bekannten Methoden einen endlichen Automaten A mit $L(A) = L(R)$.
\newline
\textit{Hinweiß: Kein Tupel angeben. Keine Minimierung vornehmen.}
\newline
\newline
\begin{tikzpicture}[auto, thick, auto]
	\node[initial, state] (S0) {$S_0$};
	\node[state, right=of S0](S1) {$S_1$};
	\node[initial, state, below= of S0](P0) {$P_0$};
	\node[state, right=of P0](P1) {$P_1$};
	\node[initial, state, below= of P0](N0) {$N_0$};
	\node[state, right=of N0](N1) {$N_1$};
\path[->]
(S0) edge node {a}(S1)
(P0) edge node {c}(P1)
(N0) edge node {b}(N1)
;
\end{tikzpicture}

(SoSe17 Probe)

\Aufgabe{4}

Gegeben sei der regulärer Ausdruck $R = ((a \cup c ) b )^*$. Konstruieren sie mit aus der Vorlesung bekannten Methoden einen endlichen Automaten A mit $L(A) = L(R)$.
\newline
\textit{Hinweiß: Kein Tupel angeben. Keine Minimierung vornehmen.}
\newline
\newline
\begin{tikzpicture}[auto, thick, auto]
	\node[initial, state] (S0) {$S_0$};
	\node[state, right=of S0](S1) {$S_1$};
	\node[initial, state, below= of S0](P0) {$P_0$};
	\node[state, right=of P0](P1) {$P_1$};
	\node[initial, state, below= of P0](N0) {$N_0$};
	\node[state, right=of N0](N1) {$N_1$};
\path[->]
(S0) edge node {a}(S1)
(P0) edge node {c}(P1)
(N0) edge node {b}(N1)
;
\end{tikzpicture}

(SoSe17 )

\Aufgabe{5}
Gegeben sei folgender nichtdeterministischer endlicher Automat.
$$ A = (\{x,y\},\{N_0,N_1,N_2,N_3\}, \{N_0,N_2\}, \delta \ gem"aß \ Graph, \{N_1,N_3\})$$


\begin{tikzpicture}[auto, thick, auto]
  \node[initial,state] (N0) {$N_0$};
  \node[state, right=of N0, accepting] (N1) {$N_1$};
\path[->]
(N0) edge [bend left] node {x} (N1)
(N1) edge [loop above] node {y} (N1);
\end{tikzpicture}

\begin{tikzpicture}[auto, thick, auto]
   \node[initial, state] (N2) {$N_2$};
     \node[state, right=of N2, accepting] (N3) {$N_3$};
\path[->]
(N2) edge [bend left] node {y} (N3)
(N2) edge [loop above] node {x} (N2);
\end{tikzpicture}
\\
\\
Geben sie einen regulären Ausdruck G an, sodass L(G) = L(A).\\
(SoSe17)



%%%%%%%%%%%%%%%%%%%%%%%%%%%%%%%%%%%%%%%%%%%%%%%%%%%%%%
%%%%%%%%%%%%%%%%%%%%%%%%%%%%%%%%%%%%%%%%%%%%%%%%%%%%%%
\end{document}

