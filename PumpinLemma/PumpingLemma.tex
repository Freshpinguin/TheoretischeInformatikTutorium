\documentclass[a4paper,12pt]{article}
\usepackage{fancyhdr}
\usepackage{fancyheadings}
\usepackage[ngerman,german]{babel}
\usepackage{german}
\usepackage[utf8]{inputenc}
%\usepackage[latin1]{inputenc}
\usepackage[active]{srcltx}
\usepackage{algorithm}
\usepackage[noend]{algorithmic}
\usepackage{amsmath}
\usepackage{amssymb}
\usepackage{amsthm}
\usepackage{bbm}
\usepackage{enumerate}
\usepackage{graphicx}
\usepackage{ifthen}
\usepackage{listings}
\usepackage{struktex}
\usepackage{hyperref}
\usepackage{tikz}
\usetikzlibrary{positioning,automata}

%%%%%%%%%%%%%%%%%%%%%%%%%%%%%%%%%%%%%%%%%%%%%%%%%%%%%%
%%%%%%%%%%%%%% EDIT THIS PART %%%%%%%%%%%%%%%%%%%%%%%%
%%%%%%%%%%%%%%%%%%%%%%%%%%%%%%%%%%%%%%%%%%%%%%%%%%%%%%
\newcommand{\Fach}{Theoretische Informatik}
\newcommand{\Name}{Robin Feldmann}
\newcommand{\Tutorium}{}
\newcommand{\Semester}{SS 21}
\newcommand{\KlausurLoesung }{SoSe2020} %  <-- UPDATE ME
%%%%%%%%%%%%%%%%%%%%%%%%%%%%%%%%%%%%%%%%%%%%%%%%%%%%%%
%%%%%%%%%%%%%%%%%%%%%%%%%%%%%%%%%%%%%%%%%%%%%%%%%%%%%%

\setlength{\parindent}{0em}
\topmargin -1.0cm
\oddsidemargin 0cm
\evensidemargin 0cm
\setlength{\textheight}{9.2in}
\setlength{\textwidth}{6.0in}

%%%%%%%%%%%%%%%
%% Aufgaben-COMMAND
\newcommand{\Aufgabe}[1]{
  {
  \vspace*{0.5cm}
  \textsf{\textbf{Aufgabe #1}}
  \vspace*{0.2cm}
  
  }
}

\newcommand{\Definition}[1]{
  {
  \vspace*{0.5cm}
  \textsf{\textbf{Definition #1}}
  \vspace*{0.2cm}
  
  }
}

\newcommand{\Satz}[1]{
  {
  \vspace*{0.5cm}
  \textsf{\textbf{#1}}
  \vspace*{0.2cm}
  
  }
}

%%%%%%%%%%%%%%
\hypersetup{
    pdftitle={\Fach{}: Übungsblatt \KlausurLoesung{}},
    pdfauthor={\Name},
    pdfborder={0 0 0}
}

\lstset{ %
language=java,
basicstyle=\footnotesize\tt,
showtabs=false,
tabsize=2,
captionpos=b,
breaklines=true,
extendedchars=true,
showstringspaces=false,
flexiblecolumns=true,
}

\title{Übungsblatt Deterministische Endliche Automaten}
\author{\Name{}}

\begin{document}
\thispagestyle{fancy}
\lhead{\sf \large \Fach{} \\ \small \Name{} }
\rhead{\sf \Semester{} \\  Tutorium \Tutorium{}}
\vspace*{0.2cm}
\begin{center}
\LARGE \sf \textbf{Pumping Lemma}
\end{center}
\vspace*{0.2cm}

%%%%%%%%%%%%%%%%%%%%%%%%%%%%%%%%%%%%%%%%%%%%%%%%%%%%%%
%% Insert your solutions here %%%%%%%%%%%%%%%%%%%%%%%%
%%%%%%%%%%%%%%%%%%%%%%%%%%%%%%%%%%%%%%%%%%%%%%%%%%%%%%
\Definition{Regul"are Sprache}
Sei \textit{X} ein Alphabet und \textit{L} eine Sprache "uber \textit{X}.\\
\textit{L} heißt regul"are Sprache genau dann wenn es einen endlichen Automaten \textit{A} (NEA oder DEA) gibt, mit L(\textit{A}) = \textit{L}.


\Satz{Pumping - Lemma} 
Sei \textit{L} eine regul"are Sprache "uber einem Alphabet \textit{X}. \\
Es gibt ein $n \in \mathbb{N}$, so dass
f"ur alle $x \in L$ mit  $\vert x \vert \geq n$ gilt: \\
Es gibt eine Zerlegung $x = uvw$ mit $u,v,w \in X^*$ und: \\
\begin{enumerate}[a)]
\item $x=uvw$
\item $\vert uv\vert \leq n$ oder $\vert vw\vert \leq n$
\item $\vert v \vert \geq 1$
\item $uv^iw \in L \ \forall i \in \mathbb{N}_0$
\end{enumerate}
Es gilt also:
$$ \text{L ist eine regul"are Sprache} \Rightarrow \text{Bei L gilt das Pumping-Lemma} $$
Durch negation der Aussage bekommen wir eine "aquivalente Aussage:
$$ \text{Bei L gilt das Pumping-Lemma NICHT} \Rightarrow \text{L ist KEINE regul"are Sprache} $$
Wie zeigt man, dass bei L das Pumping-Lemma nicht gilt?
Folgende Schritte sind zu tun:

\begin{enumerate}[1)]
\item Sei $p \in \mathbb{N}$ beliebig. \\ \textit{Einfach Abschreiben }
\item Finde ein Wort $x \in L \ mit \ \vert x\vert \geq p$. \\ \textit{Hier ist die Schwierigkeit das x so zu wählen, dass hinterher das v aus den Buchstaben bestehen muss, die das x aus der Sprache pumpen
\\  Wir wissen vom pumpbaren Anteil v nicht wie lange er ist und nicht wo er genau im Wort ist. Wir wissen nur er ist in den ersten p Buchstaben. Deshalb ist es oft g"unstig das Wort so zu w"ahlen, dass die ersten p Buchstaben gleich sind.}
\item Zerlege $x$ in $x=uvw$ mit $\vert uv\vert \leq p$ und $\vert v \vert \geq 1$.
\\ \textit{Einfach Abschreiben }
\item Bestimme aus welchen Buchstaben u,v,w bestehen.
\\ \textit{Da nicht klar ist wie lang das u,v,w genau ist, werden meist f"ur die L"angen von u,v,w neue Variablen eingeführt }
\item Bestimme $x_i = uv^iw$ mit den Buchstaben von 4).
\\ \textit{Einfach Abschreiben nur u,v,w ersetzen mit dem was man von 4ten hat. }
\item W"ahle i so, dass $x_i$ nicht mehr in der Sprache L ist.
\\ \textit{Kurz "uberlegen welche W"orter in der Sprache sind und welche nicht und dann meist i sehr groß oder gleich 0 w"ahlen. }
\item Begr"undung warum $x_i$ nicht in der Sprache ist.
\\ \textit{$x_i$ vergleichen mit der Definition der Sprache}
\item $\Rightarrow$ L ist nicht regul"ar
\\ \textit{Profit}

\end{enumerate}
\Aufgabe{1}
Zeige, dass die Sprache nicht regul"ar ist.
\begin{enumerate}[a)]
\item $L = \{a^nb^ma^n  \mid n,m \in \mathbb{N}\} $
\item $L = \{a^{2n}b^n \mid n \in \mathbb{N}\} $
\item $L = \{a^mb^n\mid n,m \in \mathbb{N}, m<n\} $
\item $L = \{a^mb^n\mid n,m \in \mathbb{N}, m> n\} $
\item $L = \{a^mb^nc^i \mid n,m,i \in \mathbb{N}, m+n<i \} $
\item $L = \{a^mb^nc^i \mid n,m,i \in \mathbb{N}, m+i<n \} $
\end{enumerate}


\Aufgabe{Klausuraufgaben}
Zeige, dass die Sprache nicht regul"ar ist.
\begin{enumerate}[a)]
\item
$L = \{ x^pyx^ky \mid p,k \in \mathbb{N}, k > p     \}$
\\ \\
(Angelehnt an SoSe17 )
\item
$L = \{xy^iz^k \mid i,k \in \mathbb{N}, i \geq k \} $
\\ \\
(Angelehnt an SoSe20)
\item
$L = \{ab^nau \mid n \in \mathbb{N},u\in \{c,d\}^*,  \vert u \vert \leq n + 1 \} $
\\ \\
(Angelehnt an WS19/20)
\item
$L = \{x^iy^jx^k \mid i \in \mathbb{N}_0 j,k \in \mathbb{N}, j<k \} $
\\ \\
(Angelehnt an SoSe2020 Probe)
\end{enumerate}






%%%%%%%%%%%%%%%%%%%%%%%%%%%%%%%%%%%%%%%%%%%%%%%%%%%%%%
%%%%%%%%%%%%%%%%%%%%%%%%%%%%%%%%%%%%%%%%%%%%%%%%%%%%%%
\end{document}

