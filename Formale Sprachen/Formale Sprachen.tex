\documentclass[a4paper,12pt]{article}
\usepackage{fancyhdr}
\usepackage{fancyheadings}
\usepackage[ngerman,german]{babel}
\usepackage{german}
\usepackage[utf8]{inputenc}
%\usepackage[latin1]{inputenc}
\usepackage[active]{srcltx}
\usepackage{algorithm}
\usepackage[noend]{algorithmic}
\usepackage{amsmath}
\usepackage{amssymb}
\usepackage{amsthm}
\usepackage{bbm}
\usepackage{enumerate}
\usepackage{graphicx}
\usepackage{ifthen}
\usepackage{listings}
\usepackage{struktex}
\usepackage{hyperref}
\usepackage{tikz}
\usetikzlibrary{positioning,automata}

%%%%%%%%%%%%%%%%%%%%%%%%%%%%%%%%%%%%%%%%%%%%%%%%%%%%%%
%%%%%%%%%%%%%% EDIT THIS PART %%%%%%%%%%%%%%%%%%%%%%%%
%%%%%%%%%%%%%%%%%%%%%%%%%%%%%%%%%%%%%%%%%%%%%%%%%%%%%%
\newcommand{\Fach}{Theoretische Informatik}
\newcommand{\Name}{Robin Feldmann}
\newcommand{\Tutorium}{}
\newcommand{\Semester}{OHM}
\newcommand{\KlausurLoesung }{SoSe2020} %  <-- UPDATE ME
%%%%%%%%%%%%%%%%%%%%%%%%%%%%%%%%%%%%%%%%%%%%%%%%%%%%%%
%%%%%%%%%%%%%%%%%%%%%%%%%%%%%%%%%%%%%%%%%%%%%%%%%%%%%%

\setlength{\parindent}{0em}
\topmargin -1.0cm
\oddsidemargin 0cm
\evensidemargin 0cm
\setlength{\textheight}{9.2in}
\setlength{\textwidth}{6.0in}

%%%%%%%%%%%%%%%
%% Aufgaben-COMMAND
\newcommand{\Aufgabe}[1]{
  {
  \vspace*{0.5cm}
  \textsf{\textbf{Aufgabe #1}}
  \vspace*{0.2cm}
  
  }
}

\newcommand{\Definition}[1]{
  {
  \vspace*{0.5cm}
  \textsf{\textbf{#1}}
  \vspace*{0.2cm}
  
  }
}

%%%%%%%%%%%%%%
\hypersetup{
    pdftitle={\Fach{}: Übungsblatt \KlausurLoesung{}},
    pdfauthor={\Name},
    pdfborder={0 0 0}
}

\lstset{ %
language=java,
basicstyle=\footnotesize\tt,
showtabs=false,
tabsize=2,
captionpos=b,
breaklines=true,
extendedchars=true,
showstringspaces=false,
flexiblecolumns=true,
}

\title{Übungsblatt Deterministische Endliche Automaten}
\author{\Name{}}

\begin{document}
\thispagestyle{fancy}
\lhead{\sf \large \Fach{} \\ \small \Name{} }
\rhead{\sf \Semester{} \\  Tutorium \Tutorium{}}
\vspace*{0.2cm}
\begin{center}
\LARGE \sf \textbf{Formale Sprachen}
\end{center}
\vspace*{0.2cm}

%%%%%%%%%%%%%%%%%%%%%%%%%%%%%%%%%%%%%%%%%%%%%%%%%%%%%%
%% Insert your solutions here %%%%%%%%%%%%%%%%%%%%%%%%
%%%%%%%%%%%%%%%%%%%%%%%%%%%%%%%%%%%%%%%%%%%%%%%%%%%%%%
\Definition{Definition}
Eine \textbf{Grammatik} ist ein Tupel $(N,T,S,P)$ wobei 
\begin{align*}
N:&\ Alphabet\  der \textbf{ nichtterminalen }Symbole \\
T:&\ Alphabet\  der \textbf{ terminalen }Symbole \text{ (mit }T\cap N=\emptyset)\\
S:&\ Startsymbol\ \in N \\
P:&\ Produktionen \ \subset (N\cup T)^+ \times(N\cup T)^*
\end{align*}
\Definition{Chomsky Hierarchie}
\begin{enumerate}[ ]
\item Typ 0: Keine Bedingung
\item Typ 1: Für alle Produktionen  $\alpha \rightarrow \beta$ gilt: $\alpha, \beta \in (N\cup T)^+$ und $\vert \alpha \vert \leq \vert \beta \vert$
\item Typ 2: Für alle Produktionen  $\alpha \rightarrow \beta$ gilt: $ \beta \in (N\cup T)^+$ und $\alpha \  \in \ N$
\item Typ 3: Für alle Produktionen  $\alpha \rightarrow \beta$ gilt: $\alpha \  \in \ N$ und $\beta = tB$, wobei $t \in T^*$ und $B \in N \cup\{ \epsilon \}$ und $\beta \neq \epsilon$.
\end{enumerate}
Sonderregel Leeres Wort:\\
Zusätlich wird die Produktion $$S_{neu} \rightarrow \epsilon \vert S_{alt}$$ erlaubt um das Leere Wort zuzulassen.
\\

\Definition{Normalformen}

\begin{tabular}{c|c|c|c|c}
Typ & 3 & 2 & 1 & 0   \\
\hline
$A\rightarrow \epsilon$ & & & &$\times$\\
\hline
$A\rightarrow t$ &$\times$ &$\times$ &$\times$ &$\times$\\
\hline
$A\rightarrow tB$ &$\times$ & & &$\times$\\
\hline
$A\rightarrow BC$ & &$\times$ &$\times$ &$\times$\\
\hline
$AB\rightarrow CD$ & & & $\times$&$\times$\\
\end{tabular}
\\




\Aufgabe{1}
Sei L=$\{(abc)^n d^m \vert n \in \mathbb{N}, \ m\in \mathbb{N}_0 \}$
\begin{enumerate}[a)]
\item Geben sie eine Typ-3 Grammatik an, die L erzeugt.
\item Geben sie auf Basis der Grammatik von (a eine Ableitung des Wortes $abcabcddd$ an.
\item Normalisieren sie die Grammatik von a).
\item Konstruieren sie den zugehörigen endlichen Automaten.
\end{enumerate}

\Aufgabe{2}
Sei L=$\{(ab)^n(cd)^m \vert k \in \mathbb{N}, \ m\in \mathbb{N}_0 \}$
\begin{enumerate}[a)]
\item Geben sie eine Typ-3 Grammatik an, die L erzeugt.
\item Geben sie auf Basis der Grammatik von (a eine Ableitung des Wortes $abcdcdcd$ an.
\item Normalisieren sie die Grammatik von a).
\item Konstruieren sie den zugehörigen endlichen Automaten.
\end{enumerate}

\Aufgabe{3}
Sei $R = ((ba)^* \cup c ) d^* $
und L die von R erzeugte Sprache.
\begin{enumerate}[a)]
\item Geben sie eine Typ-3 Grammatik an, die L erzeugt.
\item Geben sie auf Basis der Grammatik von (a eine Ableitung des Wortes $babad$ an.
\item Normalisieren sie die Grammatik von a).
\item Konstruieren sie den zugehörigen endlichen Automaten.
\end{enumerate}


\Aufgabe{4}
Geben sie die zugehörige Typ-3 Grammatik des folgenden Automaten an.


\begin{tikzpicture}[auto, thick, auto]
  \node[initial,state] (N0) {$N_0$};
  \node[state, right=of N0, accepting] (N1) {$N_1$};
\path[->]
(N0) edge [bend left] node {x} (N1)
(N1) edge [loop above] node {y} (N1);
\end{tikzpicture}

\begin{tikzpicture}[auto, thick, auto]
   \node[initial, state] (N2) {$N_2$};
     \node[state, right=of N2, accepting] (N3) {$N_3$};
\path[->]
(N2) edge [bend left] node {y} (N3)
(N2) edge [loop above] node {x} (N2);
\end{tikzpicture}
\\


\Aufgabe{5}
Geben sie die zugehörige Typ-3 Grammatik des folgenden Automaten an.

\begin{tikzpicture}[auto, thick, auto]
  \node[initial,state] (S0) {$S_0$};
  \node[state, right=of S0] (S1) {$S_1$};
   \node[state, right=of S1, accepting] (S2) {$S_2$};
\path[->]
(S0) edge [bend left] node {a} (S1)
(S1) edge [bend left] node {a} (S2)
(S1) edge [bend left] node {a,b} (S0);
\end{tikzpicture}




\Aufgabe{6}
Geben sie die zugehörige Typ-3 Grammatik des folgenden Automaten an.

\begin{tikzpicture}[auto, thick, auto]
  \node[initial,state] (S0) {$S_0$};
  \node[state, right=of S0] (S1) {$S_1$};
   \node[state, right=of S1] (S2) {$S_2$};
    \node[accepting, state, right=of S2] (S3) {$S_3$};
    \node[state, below= of S0] (S4) {$S_4$};
\path[->]
(S3) edge [loop above] node {b} (S3)
(S2) edge [loop above] node {b} (S2)
(S4) edge [loop below] node {a,b} (S4)
(S1) edge node {a} (S2)
(S2) [bend left]edge node {a} (S3)
(S0) edge node {b} (S4)
(S1) edge node {b} (S4)
(S0) edge node {a} (S1)
(S3) [bend left]edge node {a} (S2);
\end{tikzpicture}





%%%%%%%%%%%%%%%%%%%%%%%%%%%%%%%%%%%%%%%%%%%%%%%%%%%%%%
%%%%%%%%%%%%%%%%%%%%%%%%%%%%%%%%%%%%%%%%%%%%%%%%%%%%%%
\end{document}

