\documentclass[a4paper,12pt]{article}
\usepackage{fancyhdr}
\usepackage{fancyheadings}
\usepackage[ngerman,german]{babel}
\usepackage{german}
\usepackage[utf8]{inputenc}
%\usepackage[latin1]{inputenc}
\usepackage[active]{srcltx}
\usepackage{algorithm}
\usepackage[noend]{algorithmic}
\usepackage{amsmath}
\usepackage{amssymb}
\usepackage{amsthm}
\usepackage{bbm}
\usepackage{enumerate}
\usepackage{graphicx}
\usepackage{ifthen}
\usepackage{listings}
\usepackage{struktex}
\usepackage{hyperref}
\usepackage{tikz}
\usetikzlibrary{positioning,automata}

%%%%%%%%%%%%%%%%%%%%%%%%%%%%%%%%%%%%%%%%%%%%%%%%%%%%%%
%%%%%%%%%%%%%% EDIT THIS PART %%%%%%%%%%%%%%%%%%%%%%%%
%%%%%%%%%%%%%%%%%%%%%%%%%%%%%%%%%%%%%%%%%%%%%%%%%%%%%%
\newcommand{\Fach}{Theoretische Informatik}
\newcommand{\Name}{Robin Feldmann}
\newcommand{\Tutorium}{}
\newcommand{\Semester}{OHM}
\newcommand{\KlausurLoesung }{SoSe2020} %  <-- UPDATE ME
%%%%%%%%%%%%%%%%%%%%%%%%%%%%%%%%%%%%%%%%%%%%%%%%%%%%%%
%%%%%%%%%%%%%%%%%%%%%%%%%%%%%%%%%%%%%%%%%%%%%%%%%%%%%%

\setlength{\parindent}{0em}
\topmargin -1.0cm
\oddsidemargin 0cm
\evensidemargin 0cm
\setlength{\textheight}{9.2in}
\setlength{\textwidth}{6.0in}

%%%%%%%%%%%%%%%
%% Aufgaben-COMMAND
\newcommand{\Aufgabe}[1]{
  {
  \vspace*{0.5cm}
  \textsf{\textbf{Aufgabe #1}}
  \vspace*{0.2cm}
  
  }
}

\newcommand{\Definition}[1]{
  {
  \vspace*{0.5cm}
  \textsf{\textbf{Definition #1}}
  \vspace*{0.2cm}
  
  }
}

%%%%%%%%%%%%%%
\hypersetup{
    pdftitle={\Fach{}: Übungsblatt \KlausurLoesung{}},
    pdfauthor={\Name},
    pdfborder={0 0 0}
}

\lstset{ %
language=java,
basicstyle=\footnotesize\tt,
showtabs=false,
tabsize=2,
captionpos=b,
breaklines=true,
extendedchars=true,
showstringspaces=false,
flexiblecolumns=true,
}

\title{Übungsblatt Kellerautomaten}
\author{\Name{}}

\begin{document}
\thispagestyle{fancy}
\lhead{\sf \large \Fach{} \\ \small \Name{} }
\rhead{\sf \Semester{} \\  Tutorium\Tutorium{}}
\vspace*{0.2cm}
\begin{center}
\LARGE \sf \textbf{Kellerautomaten}
\end{center}
\vspace*{0.2cm}

%%%%%%%%%%%%%%%%%%%%%%%%%%%%%%%%%%%%%%%%%%%%%%%%%%%%%%
%% Insert your solutions here %%%%%%%%%%%%%%%%%%%%%%%%
%%%%%%%%%%%%%%%%%%%%%%%%%%%%%%%%%%%%%%%%%%%%%%%%%%%%%%
\Definition{Nichtdeterministischer Kellerautomat}
Ein nichtdeterministischer Kellerautomat $KA = (X,K,k_0,S,s_0,\delta, F)$ besteht aus:
\begin{align*}
X&: Eingabealphabet \\
K&: Kelleralphabet \\
k_0&: Kellerstartsymbol \in K \\
S&: Zustandsmenge \\
s_0&: Startzustand \in S \\
\delta &: Zustands"ubergangsfunktion: \delta : S \times (X \cup \{\epsilon \} )  \times K \rightarrow P_{endl}(S \times K^*) \\
F&: Menge\ der \ Endzust"ande \subseteq S\\
\end{align*}

\Definition{Konfiguration eines Kellerautomaten}
Die Konfiguration eines Kellerautomaten KA ist ein Triple $(s,w,l)$:
\begin{align*}
s \in S&: Aktueller\ Zustand \\
w\in X^*&: Resteingabe \\
l \in K^*&: Wort \ auf \ dem \  Keller  \\
\end{align*}

\Aufgabe{1} 
Konstruieren sie einen Kellerautomat der die Sprache L akzeptiert. Für L =
\begin{enumerate}[a)]
\item
$$\{ a^n b^n c^m d^m \mid n,m \in \mathbb{N} \}$$
\item
$$\{ a^n b^m c^m d^n \mid n,m \in \mathbb{N} \}$$
\item
$$\{ a^n b^m a^n \mid n,m \in \mathbb{N} \}$$
\item
$$\{ a^{2n}b^n \mid n \in \mathbb{N} \}$$
\item
$$\{ a^m b^n \mid m \in \mathbb{N}, n \in \mathbb{N}_0, m>n \}$$
\item
$$\{  a^m b^n c^i \mid m \in \mathbb{N}_0, n,i \in \mathbb{N}, m+n = i\}$$
\item
$$\{  a^m  c^i  b^n\mid m \in \mathbb{N}_0, n,i \in \mathbb{N}, m+n = i\}$$
\item
$$\{  a^m b^n   c^i  d ^{m+n+k+i+j} e^k \mid m,n,i,j,k \in \mathbb{N} \}$$
\end{enumerate}
\Aufgabe{2}
Sei L = $\{a^m b^{n}c^l b^{n+2}a^{m+k}\ \vert m,n,l\in \mathbb{N}, k \in \mathbb{N}_0\}$. \\
Geben sie einen deterministischen Kellerautomaten KA an,  mit L(KA) = L.
(WiSe19/20)


\Aufgabe{3}

Konstruieren sie einen deterministischen Kellerautomaten, der die Sprache\\ L = $\{b^n a \mid n \in \mathbb{N}\} $ akzeptiert.   
(SoSe 17 Probe).

\Aufgabe{4}
Konstruieren sie einen deterministischen Kellerautomaten, der die Sprache\\ L = $\{a^n b^{2n} c^m \mid n \in \mathbb{N}, m \in \mathbb{N}_0\} $ im Endzustand akzeptiert.   

(SoSe 20).

\Aufgabe{5}

Konstruieren sie einen deterministischen Kellerautomaten, der die Sprache\\ L = $\{a^{2m} b^{n} d^n e^m f^l \mid m,n \in \mathbb{N}, l \in \mathbb{N}_0\} $  akzeptiert.   
(SoSeProbe 20).

\Aufgabe{6}

Konstruieren sie einen deterministischen Kellerautomaten, der die Sprache\\ L = $\{a^{n} b a^m \mid m,n \in \mathbb{N}; m \geq n \} $  akzeptiert.   
(SoSe 17).

%%%%%%%%%%%%%%%%%%%%%%%%%%%%%%%%%%%%%%%%%%%%%%%%%%%%%%
%%%%%%%%%%%%%%%%%%%%%%%%%%%%%%%%%%%%%%%%%%%%%%%%%%%%%%
\end{document}

