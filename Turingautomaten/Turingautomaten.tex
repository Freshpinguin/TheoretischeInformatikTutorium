\documentclass[a4paper,12pt]{article}
\usepackage{fancyhdr}
\usepackage{fancyheadings}
\usepackage[ngerman,german]{babel}
\usepackage{german}
\usepackage[utf8]{inputenc}
%\usepackage[latin1]{inputenc}
\usepackage[active]{srcltx}
\usepackage{algorithm}
\usepackage[noend]{algorithmic}
\usepackage{amsmath}
\usepackage{amssymb}
\usepackage{amsthm}
\usepackage{bbm}
\usepackage{enumerate}
\usepackage{graphicx}
\usepackage{ifthen}
\usepackage{listings}
\usepackage{struktex}
\usepackage{hyperref}
\usepackage{tikz}
\usetikzlibrary{positioning,automata}

%%%%%%%%%%%%%%%%%%%%%%%%%%%%%%%%%%%%%%%%%%%%%%%%%%%%%%
%%%%%%%%%%%%%% EDIT THIS PART %%%%%%%%%%%%%%%%%%%%%%%%
%%%%%%%%%%%%%%%%%%%%%%%%%%%%%%%%%%%%%%%%%%%%%%%%%%%%%%
\newcommand{\Fach}{Theoretische Informatik}
\newcommand{\Name}{Robin Feldmann}
\newcommand{\Tutorium}{}
\newcommand{\Semester}{OHM}
\newcommand{\KlausurLoesung }{SoSe2020} %  <-- UPDATE ME
%%%%%%%%%%%%%%%%%%%%%%%%%%%%%%%%%%%%%%%%%%%%%%%%%%%%%%
%%%%%%%%%%%%%%%%%%%%%%%%%%%%%%%%%%%%%%%%%%%%%%%%%%%%%%

\setlength{\parindent}{0em}
\topmargin -1.0cm
\oddsidemargin 0cm
\evensidemargin 0cm
\setlength{\textheight}{9.2in}
\setlength{\textwidth}{6.0in}

%%%%%%%%%%%%%%%
%% Aufgaben-COMMAND
\newcommand{\Aufgabe}[1]{
  {
  \vspace*{0.5cm}
  \textsf{\textbf{Aufgabe #1}}
  \vspace*{0.2cm}
  
  }
}

\newcommand{\Definition}[1]{
  {
  \vspace*{0.5cm}
  \textsf{\textbf{Definition #1}}
  \vspace*{0.2cm}
  
  }
}

%%%%%%%%%%%%%%
\hypersetup{
    pdftitle={\Fach{}: Übungsblatt \KlausurLoesung{}},
    pdfauthor={\Name},
    pdfborder={0 0 0}
}

\lstset{ %
language=java,
basicstyle=\footnotesize\tt,
showtabs=false,
tabsize=2,
captionpos=b,
breaklines=true,
extendedchars=true,
showstringspaces=false,
flexiblecolumns=true,
}

\title{Übungsblatt Deterministische Endliche Automaten}
\author{\Name{}}

\begin{document}
\thispagestyle{fancy}
\lhead{\sf \large \Fach{} \\ \small \Name{} }
\rhead{\sf \Semester{} \\  Tutorium \Tutorium{}}
\vspace*{0.2cm}
\begin{center}
\LARGE \sf \textbf{Turingautomaten}
\end{center}
\vspace*{0.2cm}

%%%%%%%%%%%%%%%%%%%%%%%%%%%%%%%%%%%%%%%%%%%%%%%%%%%%%%
%% Insert your solutions here %%%%%%%%%%%%%%%%%%%%%%%%
%%%%%%%%%%%%%%%%%%%%%%%%%%%%%%%%%%%%%%%%%%%%%%%%%%%%%%
\Definition{Turingautomat}
Ein Turingautomat  $M = (X,B,S,s_0,\delta,s_f)$ besteht aus:
\begin{align*}
X&: Eingabealphabet \\
B&: Bandalphabet,\ X\subset B, \# \in B \setminus X \\
S&: Endliche \  Zustandsmenge\\
s_0&: Startzustand \in S \\
s_f&: Haltezustand, \notin S \\
\delta &: Zustands"ubergangsfunktion: \delta : S \times B  \rightarrow ((S \cup \{s_f\}) \times ( B \cup \{L,R \})) \cup \{\emptyset \}   \\
\end{align*}

\Aufgabe{1}
Konstruieren Sie einen Turingautomat, der für ein Wort $x \in \{c,d\}^*$ die Anzahl der Buchstaben $\vert x \vert$ berechnet und in Unärdarstellung auf das Band legt. Nach durchlaufen des Automaten, stehen nur die Anzahl der Buchstaben auf dem Band und der Lesekopf steht auf Anfang.
Es genügt den Zustandsübergangsgraphen anzugeben. (Angelehnt an SoSe2020) 


\Aufgabe{2} 
Für die Sprache $L = \{ x \in \{c,d\}^* \vert \text{ x beginnt mit c und endet mit d}\}$ soll ein Turingautomat konstruiert werden, der diese akzeptiert. Geben Sie den Zustandsübergangsgraphen an. (Angelehnt an SoSe2020 Probe).

\Aufgabe{3}
Der Turingautomat T mit folgendem Zustandsübergangsgraph, bekommt als Eingabe ''aa\$''.

\begin{tikzpicture}[auto, thick, auto]
  \node[initial,state] (S0) {$S_0$};
  \node[state, right=of S0] (S1) {$S_1$};
   \node[state, right=of S1] (S2) {$S_2$};
    \node[state, right=of S2] (S3) {$S_3$};
    \node[state, below=of S2] (S4) {$S_4$};
    \node[state, below=of S1] (S5) {$S_5$};
      \node[state, below=of S0] (S6) {$S_6$};
    \node[accepting, state, below= of S5] (Sf) {$S_f$};
\path[->]
(S1) edge [loop above] node {a/R} (S1)
(S6) edge [loop below] node {a/L} (S6)
(S0) edge [loop above] node {a/\#} (S0)
(S2) edge [loop above] node {b/R} (S2)
(S3) edge [loop right] node {b/R} (S3)
(S4) edge [loop below] node {b/L} (S4)
(S1) [bend left]edge node {\$/R} (S2)
(S2) [bend left]edge node {\#/b} (S3)
(S0) edge node {\#/R} (S1)
(S3) [bend left]edge node {\#/b} (S4)
(S4) [bend left]edge node {\$/L} (S5)
(S5) [bend left]edge node {a/L} (S6)
(S6) [bend left]edge node {\#/R} (S0)
(S5) [bend right]edge node {\#/\#} (Sf);
\end{tikzpicture}
\begin{enumerate}[a)]
\item Geben Sie die Konfigurationenfolge von der Startkonfiguration bis zum einmaligen Erreichen des Zustands $S_5$ an.
\item Mit welcher Konfiguration stoppt der Automat T bei der Eingabe ''aa\$''?.
\item Welche Sprache akzeptiert der Automat T?
\item Welche Berechnung führt der Automat T aus?
\end{enumerate}
(Angelehnt an SoSe17 Probe)\\

\Aufgabe{4}
Eine Turingmaschine M, erhält als Eingabe eine unicodierte Zahl.

\begin{tikzpicture}[auto, thick, auto]
  \node[initial,state] (A) {$A$};
   \node[state, right=of A] (B) {$B$};
   \node[state, right=of B] (C) {$C$};
   \node[state, right=of C] (D) {$D$};
    \node[state, right=of D] (E) {$E$};
    \node[state, right=of E] (F) {$F$};
    \node[state, below=of F] (G) {$G$};
     \node[state, below=of D] (H) {$H$};
     \node[accepting, state, below=of B] (Sf) {$S_f$};

\path[->]
(A)  edge [loop above] node {I/\#} (A)
(A)  edge [bend right]node {\#/R} (B)

(B) edge node {I/\#} (C)
(B) edge node {\#/I} (Sf)

(C) edge node {\#/R} (D)

(D) edge node {I/\#} (E)
(D) edge node {\#/I} (H)

(E) edge [bend right] node {\#/R} (F)
(F) edge [bend right] node {I/\#} (A)

(F) edge node {\#/I} (G)

(G) edge [loop below]node {I/L} (G)
(G) edge node {\#/I} (H)

(H) edge [loop below]node {I/L} (H)
(H) edge node {\#/I} (Sf);
\end{tikzpicture}
\begin{enumerate} [a)]
\item Welche Konfigurationen durchläuft der Automat für den Eingabewert 4?
\item Geben Sie das Automatentupel für M an.
\item Welche Sprache akzeptiert der Turingautomat M?
\item Welche Berechnung führt der Automat M aus?
\end{enumerate}
(Angelehnt an SoSe17)\\

%%%%%%%%%%%%%%%%%%%%%%%%%%%%%%%%%%%%%%%%%%%%%%%%%%%%%%
%%%%%%%%%%%%%%%%%%%%%%%%%%%%%%%%%%%%%%%%%%%%%%%%%%%%%%
\end{document}

