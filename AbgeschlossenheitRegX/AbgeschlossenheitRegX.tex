\documentclass[a4paper,12pt]{article}
\usepackage{fancyhdr}
\usepackage{fancyheadings}
\usepackage[ngerman,german]{babel}
\usepackage{german}
\usepackage[utf8]{inputenc}
%\usepackage[latin1]{inputenc}
\usepackage[active]{srcltx}
\usepackage{algorithm}
\usepackage[noend]{algorithmic}
\usepackage{amsmath}
\usepackage{amssymb}
\usepackage{amsthm}
\usepackage{bbm}
\usepackage{enumerate}
\usepackage{graphicx}
\usepackage{ifthen}
\usepackage{listings}
\usepackage{struktex}
\usepackage{hyperref}
\usepackage{tikz}
\usetikzlibrary{positioning,automata}

%%%%%%%%%%%%%%%%%%%%%%%%%%%%%%%%%%%%%%%%%%%%%%%%%%%%%%
%%%%%%%%%%%%%% EDIT THIS PART %%%%%%%%%%%%%%%%%%%%%%%%
%%%%%%%%%%%%%%%%%%%%%%%%%%%%%%%%%%%%%%%%%%%%%%%%%%%%%%
\newcommand{\Fach}{Theoretische Informatik}
\newcommand{\Name}{Robin Feldmann}
\newcommand{\Tutorium}{}
\newcommand{\Semester}{OHM}
\newcommand{\KlausurLoesung }{SoSe2020} %  <-- UPDATE ME
%%%%%%%%%%%%%%%%%%%%%%%%%%%%%%%%%%%%%%%%%%%%%%%%%%%%%%
%%%%%%%%%%%%%%%%%%%%%%%%%%%%%%%%%%%%%%%%%%%%%%%%%%%%%%

\setlength{\parindent}{0em}
\topmargin -1.0cm
\oddsidemargin 0cm
\evensidemargin 0cm
\setlength{\textheight}{9.2in}
\setlength{\textwidth}{6.0in}

%%%%%%%%%%%%%%%
%% Aufgaben-COMMAND
\newcommand{\Aufgabe}[1]{
  {
  \vspace*{0.5cm}
  \textsf{\textbf{Aufgabe #1}}
  \vspace*{0.2cm}
  
  }
}

\newcommand{\Definition}[1]{
  {
  \vspace*{0.5cm}
  \textsf{\textbf{#1}}
  \vspace*{0.2cm}
  
  }
}

%%%%%%%%%%%%%%
\hypersetup{
    pdftitle={\Fach{}: Übungsblatt \KlausurLoesung{}},
    pdfauthor={\Name},
    pdfborder={0 0 0}
}

\lstset{ %
language=java,
basicstyle=\footnotesize\tt,
showtabs=false,
tabsize=2,
captionpos=b,
breaklines=true,
extendedchars=true,
showstringspaces=false,
flexiblecolumns=true,
}

\title{Übungsblatt Deterministische Endliche Automaten}
\author{\Name{}}

\begin{document}
\thispagestyle{fancy}
\lhead{\sf \large \Fach{} \\ \small \Name{} }
\rhead{\sf \Semester{} \\  Tutorium \Tutorium{}}
\vspace*{0.2cm}
\begin{center}
\LARGE \sf \textbf{Abgeschlossenheit der Regul"aren Sprachen}
\end{center}
\vspace*{0.2cm}

%%%%%%%%%%%%%%%%%%%%%%%%%%%%%%%%%%%%%%%%%%%%%%%%%%%%%%
%% Insert your solutions here %%%%%%%%%%%%%%%%%%%%%%%%
%%%%%%%%%%%%%%%%%%%%%%%%%%%%%%%%%%%%%%%%%%%%%%%%%%%%%%
\Definition{Definition REG(X)}
Sei X ein Alphabet. REG(X) heißt die \textbf{Menge der regul"aren Sprachen "uber X}.

\Definition{Satz Abgeschlossenheit von REG(X)}
Reg(X) ist abgeschlossen bez"uglich:
\begin{enumerate}
\item Schnitten  $\cap$
\item Vereinigungen $\cup$
\item Komplementbildung $^c$
\item Komplexprodukt $\cdot$
\item Kleene Abschluss $^*$
\end{enumerate}
Was bedeutet das?
Haben wir zwei regul"are Sprachen $L_1$ und $L_2$ dann ist auch
\begin{enumerate}
\item deren Durchschnitt $L_1\cap L_2$,
\item deren Vereinigung $L_1\cup L_2$,
\item jeweils das Komplement $L_1^c := X^*-L_1$,
\item deren Komplexprodukt $L_1\cdot L_2$,
\item und jeweils der Kleene Abschluss $L_1^*$
\end{enumerate}
eine regul"are Sprache.
\\
\\
\Aufgabe{1}
Gegeben seien die folgende Endlichen Automaten
$$
A_1 = (\{a,b\},\{S_0,S_1,S_2,S_3\},\{S_0\},\delta_1 \ siehe\  Graph,\{S_2,S_3\})
$$
$$
A_2 = (\{a,b\},\{I_0,I_1,I_2\},\{I_0\},\delta_2 \ siehe\  Graph,\{I_2\})
$$

$A_1$\\
\begin{tikzpicture}[auto, thick, auto]
  \node[initial,state] (S0) {$S_0$};
  \node[state, right=of S0] (S1) {$S_1$};
   \node[state, right=of S1, accepting] (S2) {$S_2$};
         \node[state, below=of S1, accepting] (S3) {$S_3$};
\path[->]
(S0) edge [bend left] node {b} (S1)
(S1) edge [loop above] node {a,b} (S1)
(S3) edge [bend left] node {a} (S1)
(S2) edge [bend right] node {a,b} (S1)
(S0) edge [bend right] node {a} (S3)
(S3) edge [bend right] node {b} (S2)


;
\end{tikzpicture}
\\
$A_2$
\\
\begin{tikzpicture}[auto, thick, auto]
  \node[initial,state] (I0) {$I_0$};
   \node[state, below=of I0] (I1) {$I_1$};
    \node[state, right=of I0, accepting] (I2) {$I_2$};
\path[->]
(I0) edge [bend left] node {a} (I2)
(I0) edge [bend right] node {b} (I1)
(I1) edge [] node {a,b} (I2)
(I2) edge [bend left] node {a,b} (I1)
;

\end{tikzpicture}
\\
Konstruieren sie mit den aus der Vorlesung bekannten Verfahren folgende Automaten:
\begin{enumerate}[a)]
\item $A_1\cup A_2$,
\item $A_1^c$,
\item $A_1\cdot A_2$ und
\item $A_1^*$.
\end{enumerate}
 


\Aufgabe{2}
Gegeben seien die folgende Endlichen Automaten
$$
A_1 = (\{a,b\},\{S_0,S_1,S_2,S_3,S_4,S_5\},\{S_0\},\delta_1 \ siehe\  Graph,\{S_4,S_5\})
$$
$$
A_2 = (\{a,b\},\{I_0,I_1,I_2\},\{I_0\},\delta_2 \ siehe\  Graph,\{I_2\})
$$

$A_1$\\
\begin{tikzpicture}[auto, thick, auto]
  \node[initial,state] (S0) {$S_0$};
  \node[state, right=of S0] (S1) {$S_1$};
   \node[state, right=of S1] (S2) {$S_2$};
      \node[state, right=of S2,accepting] (S4) {$S_4$};
         \node[state, below=of S0] (S3) {$S_3$};
            \node[state, below=of S4, accepting] (S5) {$S_5$};
\path[->]
(S0) edge [bend left] node {a} (S1)
(S1) edge [bend left] node {a} (S2)
(S2) edge [bend left] node {b} (S1)
(S2) edge [bend left] node {a} (S4)
(S0) edge [bend left] node {b} (S3)
(S4) edge [bend left] node {a} (S3)
(S5) edge [bend left] node {a,b} (S3)
(S4) edge [bend left] node {b} (S5)

;
\end{tikzpicture}
\\
$A_2$
\\
\begin{tikzpicture}[auto, thick, auto]
  \node[initial,state] (I0) {$I_0$};
   \node[state, below=of I0] (I1) {$I_1$};
    \node[state, right=of I0, accepting] (I2) {$I_2$};
\path[->]
(I2) edge [loop above] node {a} (I2)
(I2) edge [bend right] node {b} (I1)
(I0) edge [bend left] node {a} (I2)
(I1) edge [bend right] node {a} (I2)
(I1) edge [loop below] node {b} (I1)
(I0) edge [bend right] node {b} (I1);

\end{tikzpicture}
\\
Konstruieren sie mit den aus der Vorlesung bekannten Verfahren folgende Automaten:
\begin{enumerate}[a)]
\item $A_1\cup A_2$,
\item $A_1^c$,
\item $A_1\cdot A_2$ und
\item $A_1^*$.
\end{enumerate}
 

\Aufgabe{3 ("Ahnlich wie Klausuraufgaben)}
\begin{enumerate}[a)]
\item
Gegeben seien die folgende Endlichen Automaten
$$
A_1 = (\{x,y,z\},\{N_0,N_1,N_2\},\{N_0\},\delta_1 \ siehe\  Graph,\{N_1,N_2\})
$$
$$
A_2 = (\{x,y,z\},\{I_0,I_1\},\{I_0\},\delta_2 \ siehe\  Graph,\{I_1\})
$$

$A_1$\\
\begin{tikzpicture}[auto, thick, auto]
  \node[initial,state] (S0) {$N_0$};
  \node[state, right=of S0, accepting] (S1) {$N_1$};
   \node[state, right=of S1, accepting] (S2) {$N_2$};
\path[->]
(S0) edge [bend left] node {x} (S1)
(S1) edge [bend left] node {y} (S2);

\end{tikzpicture}
\\
$A_2$
\\
\begin{tikzpicture}[auto, thick, auto]
  \node[initial,state] (S0) {$I_0$};
   \node[state, right=of S1, accepting] (S1) {$I_1$};
\path[->]
(S0) edge [bend left] node {z} (S1);
\end{tikzpicture}
\\
Konstruieren sie den endlichen Automaten $(A_1^* \cdot A_2)$ mit den Mitteln die sie aus der Vorlesung kennen.
\\
\item
Gegeben seien die folgende Endlichen Automaten
$$
A_1 = (\{x,y,z\},\{N_0,N_1,N_2,N_3\},\{N_0,N_2\},\delta_1 \ siehe\  Graph,\{N_1,N_2\})
$$
$$
A_2 = (\{x,y,z\},\{I_0,I_1\},\{I_0\},\delta_2 \ siehe\  Graph,\{I_1\})
$$

$A_1$\\
\begin{tikzpicture}[auto, thick, auto]
  \node[initial,state] (S0) {$N_0$};
  \node[state, right=of S0, accepting] (S1) {$N_1$};
\path[->]
(S0) edge [bend left] node {x} (S1)
(S1) edge [loop above] node {x} (S1);
\end{tikzpicture}
\begin{tikzpicture}[auto, thick, auto]
  \node[initial,state, accepting] (S0) {$N_2$};
  \node[state, right=of S0] (S1) {$N_3$};
\path[->]
(S0) edge [bend left] node {y} (S1)
(S1) edge [bend left] node {y} (S0);
\end{tikzpicture}
\\
$A_2$
\\
\begin{tikzpicture}[auto, thick, auto]
  \node[initial,state] (S0) {$I_0$};
   \node[state, right=of S1, accepting] (S1) {$I_1$};
\path[->]
(S0) edge [loop above] node {z} (S0)
(S0) edge [bend left] node {z} (S1);
\end{tikzpicture}
\\
Konstruieren sie den endlichen Automaten $(A_1 \cdot A_2)^*$ mit den Mitteln die sie aus der Vorlesung kennen.
\\


\end{enumerate}



%%%%%%%%%%%%%%%%%%%%%%%%%%%%%%%%%%%%%%%%%%%%%%%%%%%%%%
%%%%%%%%%%%%%%%%%%%%%%%%%%%%%%%%%%%%%%%%%%%%%%%%%%%%%%
\end{document}

